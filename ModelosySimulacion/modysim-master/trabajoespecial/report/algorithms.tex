\section{Algoritmo y descripción de las variables}

  \par En la siguiente sección se presentarán los algoritmos utilizados para la realización de las simulaciones del
  problema de la lavandería. Dichos algoritmos serán explicados con palabras para facilitar la comprensión del lector
  y luego se proveerá un pseudocódigo de los mismos.

  \vspace{5mm}
  \par El siguiente algoritmo simula el sistema de la lavandería, con $N$ máquinas en funcionamiento, $S$ máquinas de
  repuesto y $Op$ operarios. Para poder resolver los problemas presentados en la introducción, debemos tomar los
  siguientes valores:

  \begin{itemize}
    \item Problema 1: N = 5, S = 2, Op = 1
    \item Problema 2: N = 5, S = 2, Op = 2
    \item Problema 3: N = 5, S = 3, Op = 1
  \end{itemize}

  \par Dentro del algoritmo antes mencionado, se utilizan las siguientes variables, las cuales son inicializadas al
  comienzo del mismo:
  \begin{itemize}
    \item $time$: Variable de tiempo actual.
    \item $machines\_down\_number$: Cantidad de máquinas rotas.
    \item $repair\_time$: Lista con los tiempos de reparación de cada máquina.
    \item $life\_times$: Lista con los tiempos de vida de cada máquina, es decir, el tiempo hasta presentar una falla.
  \end{itemize}

  \par El estado del sistema puede cambiar en dos posibles casos: se averió una máquina y debe ser reemplazada, o bien,
  se terminó de reparar una máquina y la misma puede volver a ser utilizada. Para poder saber cual de dichos eventos
  ocurrió necesitamos mantener como invariante el orden de las listas $repair\_time$ y $life\_times$. A continuación
  serán explicados los dos posibles sucesos.

  \par Para resolver el primer incidente basta con verificar que el tiempo de vida de la primera máquina sea menor al
  primer tiempo de reparación, y de ser así, almacenar en la variable $time$, el tiempo de falla de la primera máquina,
  aumentando la cantidad de máquinas rotas en uno (dado que otra máquina falló). Si no existiesen máquinas de repuesto
  se terminará la simulación, puesto que el sistema no puede continuar. En caso de que existan, se genera una variable
  aleatoria $X \sim \mathcal{E}(1/T_F)$, la misma será el tiempo de falla de la nueva máquina, luego se agrega el valor
  $time + X$ a la lista $life\_time$ y se reordena dicha lista. Podría ocurrir que exista un operario libre entonces se
  le encarga la reparación de la máquina en forma inmediata, para ello se genera una variable aleatoria
  $Y \sim \mathcal{E}(1/T_R)$ y se agrega el tiempo $time + Y$ a la lista $repair\_time$, reordenando la
  misma.

  \par En el segundo incidente, se terminó de reparar una máquina por lo tanto se decrementa en uno
  $machines\_down\_number$. Además la variable $time$ será el tiempo de reparación de la primera máquina. Por otro
  lado, se calcula la cantidad de operadores libres, en caso de que hubiesen máquinas descompuestas y operarios libres
  se designa a los trabajadores dichas máquinas. Para ello, igual que en el caso anterior, se generan variables
  aleatorias $Y \sim \mathcal{E}(1/T_R)$ para cada máquina a reparar. Por último, se reordenan los valores en
  $repair\_time$ nuevamente.


  \begin{algorithm} [H]
    \caption{Experimento con $N$ máquinas en funcionamiento, $S$ de repuesto y $Op$ operarios}
    \label{alg1}
    \begin{algorithmic} [1]
    \STATE $ time \leftarrow  0 $ \label{line:1}\hfill\COMMENT{Actual time}
    \STATE $ machines\_down\_number \leftarrow  0 $ \label{line:2}
    \STATE $ repair\_time \leftarrow  \{\infty, \infty, \dots, \infty_{Op}\} $ \label{line:3}
    \STATE $ life\_times \leftarrow \{ X_1, X_2, \dots , X_N \} $ \label{line:4}
    \hfill\COMMENT{$ X_i \sim \mathcal{E}(1/T_F) $ y $ X_1 < X_2 < \dots < X_N $}

    \WHILE{System is working}
      \IF{$ life\_times_1 < repair\_time_1 $} \label{line:6}
        \RESET $ time \leftarrow  life\_times_{1}$ \label{line:7}
        \STATE $ machines\_down\_number \leftarrow  machines\_down\_number + 1 $ \label{line:8}

        \IF{$ machines\_down\_number = S + 1$}
          \RETURN $ time $ \label{line:10}
        \ENDIF
        \IF{$ machines\_down\_number < S + 1 $}
          \STATE $ X \leftarrow time + \mathcal{E}(1/T_F) $
          \STATE $ life\_times \leftarrow \{X_2, \dots, X_N, X\} $
          \STATE Reorder the values ${ X_2, \dots, X_N, X} $
        \ENDIF
        \IF{$ operator\ not\ working $}
          \STATE $ Y \leftarrow time + \mathcal{E}(1/T_F) $
          \STATE $ repair\_time \leftarrow \{X_1, \dots,\ Y, \dots, X_N, Y\} $
        \ENDIF

      \ELSE
        \RESET $ time \leftarrow repair\_time_1 $
        \STATE $ repair\_time \leftarrow \{ Y_2, Y_3, \dots, Y_N, \infty \} $
        \STATE $ free\_op \leftarrow Number\ of\ free\ operator\ $
        \IF{$ free\_op \leq machines\_down\_number $}
            \STATE $ repair\_times \leftarrow \{ time + Y_1, time + Y_2, \dots, time + Y_N \} $
                \hfill\COMMENT{$ Y_i \sim \mathcal{E}(1/T_r)$}
            \STATE $ free\_op \leftarrow 0 $
        \ENDIF
        \IF{$ machines\_down\_number = 0 $}
          \STATE $ repair\_times \leftarrow \{ \infty, \infty, \dots, \infty \} $
        \ENDIF
        \IF{$ 0 < machines\_down\_number < free\_op $}
          \STATE $ repair\_times \leftarrow \{ time + Y_1,\dots, time + Y_{free\_op}, \infty,\dots\} $
          \hfill\COMMENT{$ Y_i \sim \mathcal{E}(1/T_r)$}

        \ENDIF
        \STATE Reorder the values in $ repair\_time $
      \ENDIF
    \ENDWHILE
    \end{algorithmic}
  \end{algorithm}

  \pagebreak
  \par El algoritmo posterior se encarga de simular el experimento propiamente dicho, es decir, recrear la simulación
  la cantidad de veces necesarias para obtener los valores de la media y varianza muestrales de los tiempos de vida del
  sistema, con un nivel de confianza $\alpha$.

  \begin{algorithm}
  \caption{Simulación del experimento con confianza $\alpha$}
    \label{alg2}
    \begin{algorithmic}
      \STATE Elegir $\alpha$ como valor aceptable de $\sigma / \sqrt{n}$
      \STATE Generar al menos 1000 valores de X.
      \WHILE{$S(n) /\sqrt{n} > \alpha$}
        \STATE $n \leftarrow n + 1$
        \STATE Simular $X$ con el experimento
      \ENDWHILE
      \RETURN $\overline{X}(n)$
    \end{algorithmic}
  \end{algorithm}
