\section{\textbf{Introducción}}

	\par Un lavadero de ropa automático, cuenta con \textbf{N} máquinas lavadoras en servicio y con \textbf{S} de
	repuesto, todas ellas de idéntica marca, modelo y antigüedad. Además el lavadero cuenta con los servicios de técnicos
	que reparan las máquinas. Obviamente, los técnicos reparan las máquinas en serie, encargándose de una sola por
	vez. El problema consiste en determinar el tiempo medio (\textit{y su correspondiente desviación estándar}) que
	transcurre hasta que el lavadero deja de ser operativo (\textit{fallo del sistema}), esto es, el momento en el que se
	tiene menos de \textbf{N} máquinas en servicio, o lo que es lo mismo, posee más de \textbf{S} máquinas defectuosas en
	el taller.

	\par Todos los tiempos de funcionamiento de las máquinas, hasta descomponerse son variables independientes
	exponenciales con un tiempo medio, hasta fallar, de \textbf{$T_{F}$} y el tiempo de reparación de una máquina que
	ingresa al taller es una variable exponencial con media igual a \textbf{$T_{R}$}, independiente de todos los
	anteriores.

	\par Se analizan dos posibles mejoras al sistema de la lavandería, en una de ellas se considera el caso en el cual el
	lavadero tiene a su disposición un operario más que al principio, y otro donde se cuenta con una máquina de repuesto
	extra.

	\par El procedimiento para poder solucionar este problema consiste de una simulación mediante un algoritmo, el cual,
	por medio de los tiempos de falla y reparación de las máquinas, realiza una recopilación de datos para luego estimar
	la media y la desviación estándar muestrales. Posterior a las estimaciones, se realizan gráficos con los datos
	recolectados y por medio de los mismos se decide que tan precisas fueron las simulaciones.
