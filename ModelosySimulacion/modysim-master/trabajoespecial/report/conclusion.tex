\section{Conclusión}

  \par Partiendo del problema presentado en la introducción de este informe, se analizaron dos posibles mejoras sobre
  el sistema inicial. Las mismas consistieron, por un lado, en el agregado de una máquina de repuesto, mientras que por
  el otro, el yuxtapuesto de un operario.

  \par Luego de simular las mejoras recientemente mencionadas, podemos concluír que añadir un operario mejora
  escuetamente el funcionamiento esperado del sistema. En lo que respecta a números, la esperanza de vida del mismo
  aumenta en un tiempo cercano al mes. Mientras que lo mismo ocurre con las frecuencias relativas del sistema simulado,
  por lo que la curva en su totalidad muestra un perfeccionamiento respecto al sistema inicial.

  \par Por otro lado, se observa que resulta más conveniente añadir una lavadora de repuesto sobre agregar un operario,
  puesto que la esperanza de un sistema con dos operarios es menor que la de uno que cuenta con tres máquinas de
  respuesto. Refiriendonos a tiempo, las esperanzas difieren aproximadamente en un tiempo de dos meses. Además, podemos
  observar que las frecuencias relativas se concentran con mayor reiteración cercanas a la media, en el sistema con dos
  operarios, mientras que se distribuyen uniformemente a lo largo de la recta en el sistema con tres máquinas de
  reserva. Luego, en las simulaciones realizadas con un operario y dos o tres máquinas de repuesto, se apreciaron
  resultados similares a los ya contemplados con anterioridad.

  \par Finalmente, como apreciación personal, consideramos que este tipo de simulaciones son muy beneficiosas, ya que
  posibilitan la modelización de posibles modificaciones a sistemas como el de la lavandería. Esto permite observar
  como reacciona el sistema frente a dichas modificaciones para que luego, a partir de estos resultados se tomen las
  elecciones que mejor se adecuén a los objetivos propuestos.
